\documentclass[12pt]{article}
\usepackage[brazilian]{babel}
\usepackage[bottom=2.0cm,top=2.0cm,left=2.0cm,right=2.0cm]{geometry}
%\usepackage{fontspec}
\usepackage{indentfirst}
\usepackage{hyperref}
\usepackage{listings}
\usepackage{graphicx}
\usepackage{tabularray}
\usepackage{float}
\usepackage{amsmath}
\usepackage[oldvoltagedirection,siunitx,american]{circuitikz}

%\AddToHook{cmd/section/before}{\clearpage}
\setkeys{Gin}{width=\linewidth}

\linespread{1.25}
\parindent=1.25cm

\renewcommand{\lstlistingname}{Código}
\renewcommand{\lstlistlistingname}{Lista de códigos}
\lstset{
  language=Python,
  frame=single,
  framerule=0pt,
  framextopmargin=3ex,
  framexbottommargin=3ex,
  framexleftmargin=1em,
  xleftmargin={\dimexpr 1em+3pt},
  breaklines=true,
}

\hypersetup{colorlinks,citecolor=black,filecolor=black,linkcolor=black,urlcolor=black}

\title{Projeto de circuitos elétricos usando algoritmo genético}
\author{Jaedson Barbosa Serafim}
\date{\today}

\makeatletter
\renewcommand*{\fps@figure}{H}
\renewcommand*{\fps@table}{H}
\makeatother

\begin{document}

\maketitle

\tableofcontents

\section{Introdução}

Neste trabalho foi feito o cálculo dos valores de resistências para a máxima transferência de energia para o resistor $R_L$ no circuito \ref{fig:circuito} usando algoritmo genético.

\begin{figure}
    \centering
    \begin{circuitikz}
        \draw (0,0)
        to[V=$V_{in}$] (0,2)
        to[R=$R_1$] (2,2)
        to[R, l_=$R_2$, *-*] (2,0) -- (0,0)
        (2,2) -- (4,2)
        to[R=$R_L$] (4,0) -- (2,0)
        (3, 2) to [open,v=$V_1$] (3, 0);
    \end{circuitikz}
    \caption{Circuito em análise.}
    \label{fig:circuito}
\end{figure}

A vantagem deste circuito é que, dada a sua simplicidade, é possível facilmente saber qual a resposta para solução do problema. A máxima transferência de energia ocorre quando o resistor $R_1$ é substituído por um curto, que pode ser interpretado como uma resistência muito pequena, enquanto o resistor $R_2$ é substituído por um aberto, interpretável como uma resistência muito grande, obtendo assim o circuito equivalente \ref{fig:circuito-eq}.

\begin{figure}
    \centering
    \begin{circuitikz}
        \draw (0,0)
        to[V=$V_{in}$] (0,2) -- (2,2)
        to[R=$R_L$] (2,0) -- (0,0);
    \end{circuitikz}
    \caption{Circuito de máxima transferência de energia.}
    \label{fig:circuito-eq}
\end{figure}

A potência dissipada pelo resistor $R_L$ no circuito equivalente \ref{fig:circuito-eq} pode ser calculada usando a equação \ref{eq:formula_simplificada}.

\begin{equation}
    \label{eq:formula_simplificada}
    P_{max} = \frac{V_{in}^2}{R_L}
\end{equation}

Substituindo $V_{in}$ por $10V$ e $R_L$ por $50\Omega$ na equação \ref{eq:formula_simplificada} chegamos à máxima potência teórica para  resistência $R_L$ no circuito \ref{fig:circuito}:

\begin{equation}
    \label{eq:potencia_teorica}
    P_{max} = \frac{10^2}{50} = 2W
\end{equation}

Nessa simplificação da figura \ref{fig:circuito-eq}, como não há outros elementos resistivos no circuito, toda a potência fornecida pela fonte é entregue à resistência, ou seja, o rendimento do circuito é de 100\%.

Porém, e se não soubéssemos como fazer tais contas, e se sequer soubéssemos que circuito é esse? Nesse caso poderia ser feita essa busca pelos valores de $R_1$ e $R_2$ usando algoritmos genéticos e esse é o objetivo deste trabalho: projetar elementos de um circuito para atender a uma especificação.

\section{Função de rendimento}
\label{cap:equacionament}

A função da potência de saída do circuito \ref{fig:circuito} pode ser definida como sendo:

\begin{equation}
    \label{eq:pout}
    P_{out} = \frac{V_1^2}{R_L}
\end{equation}

$V_1$ pode ser escrito em função de $V_{in}$ e das resistências:

\begin{equation}
    \label{eq:v1}
    \begin{split}
        V_1 & = V_{in} * \frac{R_2\parallel R_L}{R_1 + R_2\parallel R_L} \\
        & = V_{in} * \frac{1}{1 + \frac{R1}{R_2\parallel R_L}} \\
        & = V_{in} * \frac{1}{1 + \frac{R1}{\frac{R_2 * R_L}{R_2 + R_L} }} \\
        & = V_{in} * \frac{1}{1 + R1 * \frac{R_2 + R_L}{R_2 * R_L} }
    \end{split}
\end{equation}

Por fim, a potência de entrada do circuito \ref{fig:circuito} é definida por:

\begin{equation}
    \label{eq:pin}
    P_{in} = V_{in} * \frac{V_{in} - V_1}{R_1}
\end{equation}

Usando as equações \ref{eq:pout}, \ref{eq:v1} e \ref{eq:pin} é possível então calcular a eficiência do circuito, ou seja, a relação entre a potência de saída e a potência de entrada, dada pela fórmula:

\begin{equation}
    \label{eq:n}
    \eta = \frac{P_{out}}{P_{in}}
\end{equation}

\section{Análise inicial}

Existem resistores com diversos valores de resistência, mas aqui a busca foi limitada de $1m\Omega$ a $100k\Omega$, que é um intervalo de busca bem abrangente caso fosse usada uma escala linear, por isso foi adotada uma escala logarítmica, para reduzir o intervalo de busca para $\left[-2,5\right]$.



Usando como base as fórmulas do capítulo \ref{cap:equacionament} foi escrito o código .

\end{document}